\documentclass[a4paper]{article}

%% Language and font encodings
\usepackage[english]{babel}
\usepackage[utf8x]{inputenc}
\usepackage[T1]{fontenc}
\usepackage{graphicx} 

%% Sets page size and margins
\usepackage[a4paper,top=3cm,bottom=2cm,left=3cm,right=3cm,marginparwidth=1.75cm]{geometry}

%% Useful packages
\usepackage{amsmath}
\usepackage{graphicx}
\usepackage[colorinlistoftodos]{todonotes}
\usepackage[colorlinks=true, allcolors=blue]{hyperref}

\title{Homework 9}
\author{Adam Karl}

\begin{document}
\maketitle

\section{Problem 27 (4 points)}
\subsection{SingleFixedHamiltonianPath $\leq$ DoubleFixedHamiltonialPath}
To transform the input for SingleFixedHamiltonianPath to an input for DoubleFixedHamiltonialPath:

\begin{itemize}
    \item Keep the same s
    \item Create a new vertex t, and add edges from t to every other vertex in G
\end{itemize}

Then run the algorithm for DoubleFixedHamiltonialPath on the graph and return whatever it returns. DoubleFixedHamiltonialPath will return true if and only if the original graph (before modification) had a hamiltonian path starting from s.

Since the transforms run in polytime, SingleFixedHamiltonianPath $\leq$ DoubleFixedHamiltonialPath.

\subsection{DoubleFixedHamiltonianPath $\leq$ SingleFixedHamiltonialPath}
To transform the input for DoubleFixedHamiltonianPath to an input for SingleFixedHamiltonialPath:

\begin{itemize}
    \item Keep the same s
    \item Create a new vertex t' with one edge connecting to t
\end{itemize}

Creating t' in this way ensures that any Hamiltonian path MUST end at t' (coming directly from t)

Then run the algorithm for SingleFixedHamiltonialPath on the graph with the original s as the start and return whatever it returns. SingleFixedHamiltonialPath will return true if and only if the original graph (before modification) had a Hamiltonian path starting from s, and ending at t.

Since the transforms run in polytime, DoubleFixedHamiltonianPath $\leq$ SingleFixedHamiltonialPath.

\subsection{SingleFixedHamiltonianPath $\leq$ HamiltonianCycle}
To transform the input for SingleFixedHamiltonianPath to an input for HamiltonianCycle:

\begin{itemize}
    \item Add edges from s to every other vertex in the graph (that it doesn't already have an edge to)
\end{itemize}

Then run the algorithm for HamiltonianCycle on the graph and return whatever it returns. HamiltonianCycle will return true if and only if the original graph (before modification) had a Hamiltonian path starting from s. Adding the edges from s to every other vertex in the graph ensures that at the end of the Hamiltonian path, the last vertex has an edge to return to s to complete the cycle

Since the transforms run in polytime, SingleFixedHamiltonianPath $\leq$ HamiltonianCycle.

\subsection{HamiltonianCycle $\leq$ SingleFixedHamiltonianPath}
To transform the input for HamiltonianCycle to an input for SingleFixedHamiltonianPath:

\begin{itemize}
    \item Choose an arbitrary vertex in the graph as s
    \item Create a new vertex t, and add edges to every vertex that s has an 
    \item Create a new vertex t', and only connect it to t.
\end{itemize}

Since t' only connects to one vertex, any valid Hamiltonian path MUST end at t'. In order to reach t', there must be a valid Hamiltonian path that starts at s, touches all other vertices before going to t, then t'. remember that s and t were split from the same vertex. Thus, running SingleFixedHamiltonianPath on this graph (with s as the start point), will return true if and only if the original graph had a Hamiltonian cycle.

Since the transforms run in polytime, HamiltonianCycle $\leq$ SingleFixedHamiltonianPath.

\subsection{Conclusion}
We have established:
\begin{itemize}
    \item SingleFixedHamiltonianPath $\leq$ DoubleFixedHamiltonialPath
    \item DoubleFixedHamiltonianPath $\leq$ SingleFixedHamiltonialPath
    \item SingleFixedHamiltonianPath $\leq$ HamiltonianCycle
    \item HamiltonianCycle $\leq$ SingleFixedHamiltonianPath
\end{itemize}

Although we have not done all 6 pairwise reductions, this set is sufficient to conclude that each of the algorithms can be reduced to each other (for instance, to prove HamiltonianCycle $\leq$ DoubleFixedHamiltonianPath we could solve HamiltonianCycle by running the SingleFixedHamiltonianPath on it, but have that SingleFixedHamiltonianPath use the reduction to DoubleFixedHamiltonialPath we showed in 1.1). 

Since all 3 problems can be reduced to each other in polytime, if we are given a polytime algorithm for any 1 of the three, we can use our reductions to find polytime algorithms for the other 2. Thus, if any one of these problems have a polynomial-time algorithm then they all have polynomial time algorithms.

\end{document}
