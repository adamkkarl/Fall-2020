\documentclass[a4paper]{article}

%% Language and font encodings
\usepackage[english]{babel}
\usepackage[utf8x]{inputenc}
\usepackage[T1]{fontenc}
\usepackage{graphicx} 

%% Sets page size and margins
\usepackage[a4paper,top=3cm,bottom=2cm,left=3cm,right=3cm,marginparwidth=1.75cm]{geometry}

%% Useful packages
\usepackage{amsmath}
\usepackage{graphicx}
\usepackage[colorinlistoftodos]{todonotes}
\usepackage[colorlinks=true, allcolors=blue]{hyperref}

\title{Homework 13}
\author{Adam Karl}

\begin{document}
\maketitle

\section{Problem 41 (8 points)}
\subsection{Motivation}
During class on Monday 11/16, we created a CREW PRAM algorithm that solves shortest path in $lg^2(n)$ time using $n^3/lg(n)$ processors. Although I will reference this algorithm and use it in my solution, I will not be re-explaining it.

With this in mind, our goal is to reduce Longest Common Subsequence (LCS) $\leq$ ShortestPath such that the input and output transformations are both $O(lg^2(n))$ with a polynomial number of processors. If this is possible (and we have at least $n^3/lg(n)$ processors as is needed to solve ShortestPath in $O(lg^2(n))$ time), then we can combine the reduction with our solution from class in order to solve LCS in $O(lg^2(n))$ time.

\subsection{Transformation Description}


\subsection{LCS $\leq$ ShortestPath}
With at least $n^2$ processors, we have shown that the input and output transformations from LCS to ShortestPath are constant time, and the result from ShortestPath can correctly solve LCS. Therefore, with at least $n^2$ processors, LCS $\leq _{constant time}$ ShortestPath. Since our ShortestPath algorithm from class runs in $lg^2(n)$ time using $n^3/lg(n)$ processors, we can utilize that algorithm along with the described input/output transformations to solve LCS in $lg^2(n)$ time using $n^3/lg(n)$ processors.

\end{document}
