\documentclass[a4paper]{article}

%% Language and font encodings
\usepackage[english]{babel}
\usepackage[utf8x]{inputenc}
\usepackage[T1]{fontenc}
\usepackage{graphicx} 

%% Sets page size and margins
\usepackage[a4paper,top=3cm,bottom=2cm,left=3cm,right=3cm,marginparwidth=1.75cm]{geometry}

%% Useful packages
\usepackage{amsmath}
\usepackage{graphicx}
\usepackage[colorinlistoftodos]{todonotes}
\usepackage[colorlinks=true, allcolors=blue]{hyperref}

\title{Homework 13}
\author{Adam Karl}

\begin{document}
\maketitle

\section{Problem 41 (8 points)}
\subsection{Motivation}
During class on Monday 11/16, we created a CREW PRAM algorithm that solves shortest path in $lg^2(n)$ time using $n^3/lg(n)$ processors. Although I will reference this algorithm and use it in my solution, I will not be re-explaining it.

With this in mind, our goal is to reduce Longest Common Subsequence (LCS) $\leq$ ShortestPath such that the input and output transformations are both $O(lg^2(n))$ with a polynomial number of processors. If this is possible (and we have at least $n^3/lg(n)$ processors as is needed to solve ShortestPath in $O(lg^2(n))$ time), then we can combine the reduction with our solution from class in order to solve LCS in $O(lg^2(n))$ time.

\subsection{Transformation Description}

To solve LCS(a,b) using an algorithm for ShortestPath, first:

\begin{itemize}
    \item Let sequences a and b have lengths n and m, such that n>m.
    \item Assume at least $nm$ processors.
\end{itemize}

then:

\begin{itemize}
    \item Create G with $nm$ vertices, labeled $v_{1 \rightarrow n+1,1 \rightarrow m+1}$
    \begin{itemize}
        \item (constant time with each of $nm$ processors creating precisely 1 vertex)
    \end{itemize}
    \item Add edges from each vertex to the one with a higher first coordinate, if any (e.g. from $V_{i,j}$ to $V_{i+1,j}$) with a weight of L
    \begin{itemize}
        \item Choose L such that it is larger than $n$. This will make it easy to extract the LCS solution from what ShortestPath returns
        \item (again, constant time since 1 processor per vertex, each processor adds 1 edge)
    \end{itemize}
    \item Add edges from each vertex to the one with a higher second coordinate, if any (e.g. from $V_{i,j}$ to $V_{i,j+1}$) with a weight of L
    \begin{itemize}
        \item (again, constant time since 1 processor per vertex, each processor adds 1 edge)
    \end{itemize}
    \item At each vertex $V_{i,j}$, compare characters a[i] to b[j] and create an edge to $V_{i+1,j+1}$ if a[i] = b[j]
    \begin{itemize}
        \item (again, constant time since 1 processor per vertex, each processor makes 2 reads (unimpeded since CR), then creates 0 or 1 edges)
    \end{itemize}
    \item Run the algorithm for ShortestPath on G with start point $V_{1,1}$ and endpoint $V_{n+1,m+1}$ and return what it returns mod L.
\end{itemize}

\subsection{Visualization}
If LCS is run on "abcde" and "aces", then the corresponding graph would look like:

\begin{center}
    \includegraphics[scale=.5]{hw13-1.png}
\end{center}

where each horizontal and vertical edge has a weight of, for instance 10, and each diagonal has a weight of 1. We then run Shortest path on the graph using $V_{1,1}$ (top left) and $V_{6,5}$ (bottom right) as the endpoints. On this input, Shortest path would return 33 as shown:

\begin{center}
    \includegraphics[scale=.5]{hw13-2.png}
\end{center}

and LCS would return 3 (33 mod 10).

LCS returns precisely the number of "diagonals" taken in the ShortestPath grid, which corresponds to the number of characters in the longest common subsequence.

\subsection{LCS $\leq$ ShortestPath}
With at least $n^2$ processors, we have shown that the input and output transformations from LCS to ShortestPath are constant time, and the result from ShortestPath can correctly solve LCS. Therefore, with at least $n^2$ processors, LCS $\leq _{constant time}$ ShortestPath. Since our ShortestPath algorithm from class runs in $lg^2(n)$ time using $n^3/lg(n)$ processors, we can utilize that algorithm along with the described input/output transformations to solve LCS in $lg^2(n)$ time using $n^3/lg(n)$ processors.

\end{document}
