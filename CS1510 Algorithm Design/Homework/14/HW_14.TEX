\documentclass[a4paper]{article}

%% Language and font encodings
\usepackage[english]{babel}
\usepackage[utf8x]{inputenc}
\usepackage[T1]{fontenc}
\usepackage{graphicx} 

%% Sets page size and margins
\usepackage[a4paper,top=3cm,bottom=2cm,left=3cm,right=3cm,marginparwidth=1.75cm]{geometry}

%% Useful packages
\usepackage{amsmath}
\usepackage{graphicx}
\usepackage[colorinlistoftodos]{todonotes}
\usepackage[colorlinks=true, allcolors=blue]{hyperref}

\title{Homework 14}
\author{Adam Karl}

\begin{document}
\maketitle

\section{Problem 45 (8 points)}
\subsection{a.}


\subsection{b. Max with priority processors}
\begin{itemize}
    \item Create an array A of length n, each index initialized to false
    \item Create an answer variable to store the maximum value in the sequence
    \item Place one processor at each number $x_i$
    \item Each processor at $x_i$ should do:
    \begin{itemize}
        \item concurrently write true to A[$x_i$]
        \item //since each processor is only writing true, it doesn't matter which one writes to A[$x_i$] if there are multiple values of $x_i$ in the sequence
    \end{itemize}
    \item Move the processors so that 1 processor is at each index of A, in descending order of priority
    \begin{itemize}
        \item //the highest priority processor is at index n, the lowest priority processor is at index 1
        \item //the runtime of this algorithm is based on the assumption that this step can be done in constant time
    \end{itemize}
    \item Each processor at A[i] should do:
    \begin{itemize}
        \item if A[i] is true, concurrently write i to the answer variable
    \end{itemize}
    \item return the value stored in the answer variable
    
\end{itemize}


\subsection{c. Max without priority processors}


\end{document}
