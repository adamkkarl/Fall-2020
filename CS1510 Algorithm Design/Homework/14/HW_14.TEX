\documentclass[a4paper]{article}

%% Language and font encodings
\usepackage[english]{babel}
\usepackage[utf8x]{inputenc}
\usepackage[T1]{fontenc}
\usepackage{graphicx} 

%% Sets page size and margins
\usepackage[a4paper,top=3cm,bottom=2cm,left=3cm,right=3cm,marginparwidth=1.75cm]{geometry}

%% Useful packages
\usepackage{amsmath}
\usepackage{graphicx}
\usepackage[colorinlistoftodos]{todonotes}
\usepackage[colorlinks=true, allcolors=blue]{hyperref}

\title{Homework 14}
\author{Adam Karl}

\begin{document}
\maketitle

\section{Problem 45 (8 points)}
\subsection{Motivation}
Problems (a) and (c) will both be partially dependent on an algorithm we discussed in class that used pairwise comparisons on a CRCW PRAM to find the maximum of n numbers in constant time with n$^2$ processors.

\subsection{a.}
To solve this problem in lg(lg(n)) time do:

\begin{itemize}
    \item let k = the number of remaining values
    \item while (k>1) do:
    \begin{itemize}
        \item group the values into k$^2$ groups, each with n/k values
        \item Run the algorithm from class on each group to find the max
        \item //since each group has n/k items, each group needs (n/k)$^2$ processors to run in constant time.
        \item //since there are k$^2$/n groups, the total number of processors needed to complete this iteration in constant time is (n/k)$^2$ * k$^2$/n = n
        \item //since we have n processors, this iteration completes in constant time.
    \end{itemize}
    \item when there is only 1 value remaining, return that value
\end{itemize}

Since each step takes constant time, our runtime is based on the number of steps that are necessary to solve a sequence of size n.

\begin{table}[htb]
\centering
\begin{tabular}{lll}
step (k) & group size (G)               & output size (I)           \\
0    & 1                            & n                         \\
1    & 2                            & n/(2\textasciicircum{}1)  \\
2    & 4 (=2\textasciicircum{}2)    & n/(2\textasciicircum{}3)  \\
3    & 16 (=4\textasciicircum{}2)   & n/(2\textasciicircum{}7)  \\
4    & 256 (=16\textasciicircum{}2) & n/(2\textasciicircum{}15)
\end{tabular}
\end{table}

The recurrence relation is:
\begin{itemize}
    \item G(2) = 2
    \item G(k) = (G(k-1))$^2$
    \item I(1) = n
    \item I(k+1) = I(k)/G(k)
\end{itemize}

With this in mind, look at the relation between number of steps and the denominator for output size. The algorithm finishes when the output size is 1, so in 1 step we can solve an input of size n=2, in 2 steps we can handle n=8, in 3 steps n=128, in 4 steps n=32768, etc. The pattern is: in k steps, we are able to handle an input of size n = $2^{(2^k)-1}$. Therefore, (since each step takes constant time) we are able to handle an input of size n in lg(lg(n)) steps (simply take the log of both sides twice). Therefore, our runtime is O(lg(lg(n))).

\subsection{b. Max with priority processors}
To solve this problem in linear time do:
\begin{itemize}
    \item Create an array A of length n, each index initialized to false
    \item Create an answer variable to store the maximum value in the sequence
    \item Place one processor at each number $x_i$
    \item Each processor at $x_i$ should do:
    \begin{itemize}
        \item concurrently write true to A[$x_i$]
        \item //since each processor is only writing true, it doesn't matter which one writes to A[$x_i$] if there are multiple values of $x_i$ in the sequence
    \end{itemize}
    \item Move the processors so that 1 processor is at each index of A, in descending order of priority
    \begin{itemize}
        \item //the highest priority processor is at index n, the lowest priority processor is at index 1
        \item //(the runtime of this algorithm is based on the assumption that this step can be done in constant time)
    \end{itemize}
    \item Each processor at A[i] should do:
    \begin{itemize}
        \item if A[i] is true, concurrently write i to the answer variable
        \item //due to the prioritization of the processors, only the processor at the maximum true index will write
    \end{itemize}
    \item return the value stored in the answer variable
    
\end{itemize}


\subsection{c. Max without priority processors}
To solve this problem in linear time do:

\begin{itemize}
    \item Create an array A of length n, each index initialized to false
    \item Create an array B of length sqrt(n), each index initialized to false
    \item Create an answer variable to store the maximum value in the sequence
    \item Place one processor at each number $x_i$
    \item Each processor at $x_i$ should do:
    \begin{itemize}
        \item concurrently write true to A[$x_i$]
        \item //since each processor is only writing true, it doesn't matter which one writes to A[$x_i$] if there are multiple values of $x_i$ in the sequence
    \end{itemize}
    \item //here marks where the algorithm deviates from the solution to (b) in the approach used to find the "right-most" true in A
    \item Place one processor at each index of A
    \item Each processor at A[i] should do:
    \begin{itemize}
        \item if A[i] is true, concurrently write true to B[$\lfloor$i/sqrt(n)$\rfloor$]
        
        \item //the idea is that processors at A[1] through A[sqrt(n)] all point to B[1]
        \item //A[sqrt(n) + 1] through A[2*sqrt(n)] all point to B[2]
        \item ...
        \item //A[n-sqrt(n)] through A[n] all point to B[sqrt(n)]
    \end{itemize}
    \item Run the algorithm from class to find the max of sqrt(n) numbers in constant time with n processors. The input is each INDEX i of B for which B[i] is true.
    \begin{itemize}
        \item //the algorithm returns which sqrt(n)-size "group" of A has the right-most true
    \end{itemize}
    \item take the solution to the algorithm (say: z) and go the the z-th group in A.
    \item Run the algorithm from class again on the z-th group of A (with all n processors) to find the right-most index of A for which A[i] is true. Return this index i as the answer.
    
\end{itemize}


\end{document}
