\documentclass[a4paper]{article}

%% Language and font encodings
\usepackage[english]{babel}
\usepackage[utf8x]{inputenc}
\usepackage[T1]{fontenc}
\usepackage{graphicx} 

%% Sets page size and margins
\usepackage[a4paper,top=3cm,bottom=2cm,left=3cm,right=3cm,marginparwidth=1.75cm]{geometry}

%% Useful packages
\usepackage{amsmath}
\usepackage{graphicx}
\usepackage[colorinlistoftodos]{todonotes}
\usepackage[colorlinks=true, allcolors=blue]{hyperref}

\title{Homework 10}
\author{Adam Karl}

\begin{document}
\maketitle

\section{Problem 31 (6 points)}
\subsection{a. clique $\leq$ A}
To solve Clique(G,k) (a known NP-hard problem) using an algorithm to solve A, do:

\begin{itemize}
    \item Count the number of vertices in G. Let this value be n.
    
    \item Create G' as a copy of G
    
    \item if $k > 3n/4$:
    \begin{itemize}
        \item add $\left\lfloor4k-3n\right\rfloor$ vertices to G'. They have no edges
        \item This value is derived from k = 3(n + numNewVertices)/4
        \item this is the case where Clique(G,k) is looking for a larger clique than A(G) would, so we have to add vertices (unconnected so that they don't change the size of the largest clique) until A(G') is looking for a clique of exactly size k.
    \end{itemize}
    
    \item else:
    \begin{itemize}
        \item add $\left\lfloor3n-4k\right\rfloor$ vertices to G', and connect each of these vertices to all other vertices in G'
        \item This value is derived from k + numNewVertices = 3(n + numNewVertices)/4
        \item this is the case where Clique(G,k) is looking for a smaller clique than A(G) would, so we have to add vertices (connected so that adding 1 new vertex increases the size of the largest clique by 1) to boost the size of the clique until B(G') is returns true if and only if Clique(G,k) should return true.
    \end{itemize}
    \item run B(G') and return what it returns
\end{itemize}

The input transformation is quadratic (in the worst case when adding connected vertices), the output transformation is constant, and A(G') returns true if and only if Clique(G,k) would return true, so Clique $\leq$ A. Since Clique is a known NP-hard problem, B must be as well.

\subsection{b. trivial reduction A $\leq$ B}
To solve A(G), take G and find the complement graph G' by having all vertices from G in G', then having 2 vertices connected by an edge if and only if they are not connected by an edge in G. Run B(G') and return what it returns.

The input transformation is quadratic, the output transformation is constant, and B(G') returns true if and only if A(G) would return true, so A $\leq$ B. Since A is also NP-hard, B must be as well.

\subsection{c. Clique $\leq$ C}
To solve Clique(G,k) (a known NP-hard problem) using an algorithm to solve C, do:

\begin{itemize}
    \item create G' as a copy of G
    \item add k vertices to G' that have no edges
    \item run C(G',k) and return what it returns
\end{itemize}

By adding k unconnected vertices to G' we ensure that the second condition (independent set of size k) of C is always satisfied without affecting the first condition (clique of size k). Running C on G' now returns true if and only if Clique(G) has a clique of size k. Thus Clique $\leq$ C.

Since Clique can be reduced to C in polytime, and Clique is a known NP-Hard problem, then C must also be NP-Hard.

\subsection{d}

\subsection{e. polytime algorithm}
A polytime algorithm for E(G) is:
\begin{itemize}
    \item return false
\end{itemize}

Since this algorithm has a polytime (constant, in fact) solution, it is part of P and cannot be NP-hard.

It is not possible for most of the vertices in a graph to be part of a clique, and for most vertices to be part of an independent set, since there must be overlap between the 2 types of vertices but ones in the clique MUST have $(3n/4) -1$ edges and those in the independent set can have at MOST $(n/4)$ edges (connected to everything except those in the independent set). It is not possible to have at least $(3n/4) -1$ edges AND at most $(n/4)$ edges.

Note: depending on the exact definition of clique and independent set, a graph with 1 vertex may satisfy both conditions. If so, it would be the only graph to return true on.

\subsection{f. A $\leq$ F}
To solve A(G) (which we already proved is NP-hard) using an algorithm to solve F, do:

\begin{itemize}
    \item create G' as a copy of G
    \item for each vertex in G':
    \begin{itemize}
        \item while the vertex has fewer than n/2 edges:
        \begin{itemize}
            \item add an arbitrary edge from connecting the vertex to another vertex (that it isn't already connected to)
        \end{itemize}
    \end{itemize}
    \item run F(G') and return what it returns
\end{itemize}

By making sure each vertex has at least n/2 edges, we ensure that the second condition of F (independent set of size 3n/4) never returns true, but since $n/2<3n/4$ adding edges this way cannot create a clique of size 3n/4 where there wasn't one before since the vertex needs (3n/4)-1 edges to possibly be in a clique of size 3n/4.

Since F will now return true if and only if A(G) should turn true, and the transformations are polytime, A is polytime reducible to F. Since we already proved A was NP-Hard, F must be as well.

Note that there is an edge case for very few (for instance: 4) vertices where adding n/2 edges CAN actually give a clique of size 3 where there wasn't one before. However, weeding out these edge cases is an implementation detail and $2n+1=3n/4$ cannot happen for larger values of n.

\end{document}
