\documentclass[a4paper]{article}

%% Language and font encodings
\usepackage[english]{babel}
\usepackage[utf8x]{inputenc}
\usepackage[T1]{fontenc}
\usepackage{graphicx} 

%% Sets page size and margins
\usepackage[a4paper,top=3cm,bottom=2cm,left=3cm,right=3cm,marginparwidth=1.75cm]{geometry}

%% Useful packages
\usepackage{amsmath}
\usepackage{graphicx}
\usepackage[colorinlistoftodos]{todonotes}
\usepackage[colorlinks=true, allcolors=blue]{hyperref}

\title{Homework 5}
\author{Adam Karl}

\begin{document}
\maketitle

\section{Problem 16 (6 points)}
\subsection{Motivation}

It is possible for an AVL tree with 7 nodes to have either height 2 or height 3. 

It is possible for an AVL tree with 14 nodes to have either height 3 or height 4. 

Therefore, imagine there are 22 total nodes and we are trying to find the best solution that has node 8 as the root, creating two subtrees one with 7 nodes and one with 14 nodes. 

Valid solutions
\begin{itemize}
    \item 7 nodes => 2 height; 14 nodes => 3 height (balance factor = 1)
    \item 7 nodes => 3 height; 14 nodes => 3 height (balance factor = 0)
    \item 7 nodes => 3 height; 14 nodes => 4 height (balance factor = 1)
\end{itemize}

Invalid solutions
\begin{itemize}
    \item 7 nodes => 2 height; 14 nodes => 2 height (balance factor = 2)
\end{itemize}
It is NOT sufficient to simply find the optimal AVL tree for each subproblem, since it is possible that the optimal subtree for 7 nodes has a height of 2 and the optimal subtree for 14 nodes has a height of 4. In this case, even though the left subtree is a valid AVL tree and the right subtree is a valid AVL tree, combining them would give the root node an invalid balance factor of 2, thus creating an invalid overall AVL tree.

Thus, when calculating optimal subtrees, we must also 

\subsection{Algorithm Description}


\end{document}
