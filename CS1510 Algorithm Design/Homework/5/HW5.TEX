\documentclass[a4paper]{article}

%% Language and font encodings
\usepackage[english]{babel}
\usepackage[utf8x]{inputenc}
\usepackage[T1]{fontenc}
\usepackage{graphicx} 

%% Sets page size and margins
\usepackage[a4paper,top=3cm,bottom=2cm,left=3cm,right=3cm,marginparwidth=1.75cm]{geometry}

%% Useful packages
\usepackage{amsmath}
\usepackage{graphicx}
\usepackage[colorinlistoftodos]{todonotes}
\usepackage[colorlinks=true, allcolors=blue]{hyperref}

\title{Homework 5}
\author{Adam Karl}

\begin{document}
\maketitle

\section{Problem 16 (6 points)}
\subsection{Motivation}

\begin{figure}
\centerline{\includegraphics{7AVL.png}}
\caption{AVL trees with 7 nodes and different heights}
\label{fig}
\end{figure}

It is possible for a valid AVL tree of n nodes to have multiple different heights.
\begin{itemize}
    \item It is possible for a valid AVL tree with 7 nodes to have either height 2 or height 3. (see figure 1)
    \item It is possible for a valid AVL tree with 14 nodes to have either height 3 or height 4. 
\end{itemize}

Therefore, imagine there are 22 total nodes and we are trying to find the best solution that has node 8 as the root. There are two subtrees, one for the 7 nodes to the left of 8 and one for the 14 nodes to the right of 8.

Valid solutions
\begin{itemize}
    \item 7 nodes => 2 height; 14 nodes => 3 height (balance factor = 1)
    \item 7 nodes => 3 height; 14 nodes => 3 height (balance factor = 0)
    \item 7 nodes => 3 height; 14 nodes => 4 height (balance factor = 1)
\end{itemize}

Invalid solutions
\begin{itemize}
    \item 7 nodes => 2 height; 14 nodes => 4 height (balance factor = 2 INVALID)
\end{itemize}
It is NOT sufficient to simply find the optimal AVL tree for each subproblem, since it is possible that the optimal subtree for 7 nodes has a height of 2 and the optimal subtree for 14 nodes has a height of 4. In this case, even though the left subtree is a valid AVL tree and the right subtree is a valid AVL tree, combining them would give the root node an invalid balance factor of 2, thus creating an invalid overall AVL tree.

Thus, when calculating optimal subtrees/subproblems, we must calculate the optimal solutions FOR EACH POSSIBLE HEIGHT. Then, we are able to only combine 2 subtrees when their heights differ by only 0 or 1.

\subsection{Algorithm Description}


(tree height, expected depth of a key, sum of probabilities for $p_i$ through $p_j$)

Base Case:
    If i=j, then set A[i,j] = (0, 0, $p_i$
    Reason: 
    
\subsection{Potential Changes}
For each element in the 2D array, we store a number of (tree height, expected depth of a key, sum of probabilities for $p_i$ through $p_j$) tuples. The third value in the tuple (sum of probabilities for $p_i$ through $p_j$) is NOT strictly necessary, as we could instead always calculate this O(n) sum every single time. Technically, this would still remain a polynomial-time algorithm, but it would be far less efficient.

Additionally, we could instead convert 2D array with an unknown number of tuples in each element to a 3D array with a single tuple (or value, if the previous change is also done). We would do this by creating a third dimension for height, where A[i, j, h] would contain the optimal expected depth of a key for the optimal AVL subtree containing trees i through j (inclusive). At the end of building the array we would check the row of A[1, N, h] over all values of h for the optimal solution. The reason I dislike this solution (and the reason I didn't implement it for my solution) is that it creates a LOT of NULL or empty elements in the 3D array, which seems wasteful of space. For instance, obviously A[1, 7, 1] would be impossible, since no AVL tree with 7 elements could have a height of 1. In fact, since an AVL tree with 7 nodes can ONLY have a height of 2 or 3, ALL other values for h other than 2 or 3 would have an empty cell at A[1, 7, h].
Note that this solution IS strictly faster than my solution (since you don't have to traverse a linked list of tuples for each array element), but it would be a tradeoff with wasted space.

When n is small, the possible AVL trees only have a couple different possible heights, so traversing a linked list with 1 entry for each height is negligible. When n is large, this change would save time, but at the cost of creating a massive (and almost entirely empty) array that wastes a ton of space.



\end{document}
