\documentclass[a4paper]{article}

%% Language and font encodings
\usepackage[english]{babel}
\usepackage[utf8x]{inputenc}
\usepackage[T1]{fontenc}
\usepackage{graphicx} 

%% Sets page size and margins
\usepackage[a4paper,top=3cm,bottom=2cm,left=3cm,right=3cm,marginparwidth=1.75cm]{geometry}

%% Useful packages
\usepackage{amsmath}
\usepackage{graphicx}
\usepackage[colorinlistoftodos]{todonotes}
\usepackage[colorlinks=true, allcolors=blue]{hyperref}

\title{Homework 5}
\author{Adam Karl}

\begin{document}
\maketitle

\section{Problem 16 (6 points)}
\subsection{Motivation}

\begin{figure}
\centerline{\includegraphics{7AVL.png}}
\caption{AVL trees with 7 nodes and different heights}
\label{fig}
\end{figure}

It is possible for a valid AVL tree of n nodes to have multiple different heights.
\begin{itemize}
    \item It is possible for a valid AVL tree with 7 nodes to have either height 2 or height 3. (see figure 1)
    \item It is possible for a valid AVL tree with 14 nodes to have either height 3 or height 4. 
\end{itemize}

Therefore, imagine there are 22 total nodes and we are trying to find the best solution that has node 8 as the root. There are two subtrees, one for the 7 nodes to the left of 8 and one for the 14 nodes to the right of 8.

Valid solutions
\begin{itemize}
    \item 7 nodes => 2 height; 14 nodes => 3 height (balance factor = 1)
    \item 7 nodes => 3 height; 14 nodes => 3 height (balance factor = 0)
    \item 7 nodes => 3 height; 14 nodes => 4 height (balance factor = 1)
\end{itemize}

Invalid solutions
\begin{itemize}
    \item 7 nodes => 2 height; 14 nodes => 4 height (balance factor = 2 INVALID)
\end{itemize}
It is NOT sufficient to simply find the optimal AVL tree for each subproblem, since it is possible that the optimal subtree for 7 nodes has a height of 2 and the optimal subtree for 14 nodes has a height of 4. In this case, even though the left subtree is a valid AVL tree and the right subtree is a valid AVL tree, combining them would give the root node an invalid balance factor of 2, thus creating an invalid overall AVL tree.

Thus, when calculating optimal subtrees/subproblems, we must calculate the optimal solutions FOR EACH POSSIBLE HEIGHT. Then, we are able to only combine 2 subtrees when their heights differ by only 0 or 1.

\subsection{Algorithm Description}


\end{document}
