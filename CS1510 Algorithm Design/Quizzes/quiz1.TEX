\documentclass[a4paper]{article}

%% Language and font encodings
\usepackage[english]{babel}
\usepackage[utf8x]{inputenc}
\usepackage[T1]{fontenc}

%% Sets page size and margins
\usepackage[a4paper,top=3cm,bottom=2cm,left=3cm,right=3cm,marginparwidth=1.75cm]{geometry}

%% Useful packages
\usepackage{amsmath}
\usepackage{graphicx}
\usepackage[colorinlistoftodos]{todonotes}
\usepackage[colorlinks=true, allcolors=blue]{hyperref}

\title{Quiz 1}
\author{Adam Karl}

\begin{document}
\maketitle


\section{Proof by Exchange Argument}

Note: for this problem I will be referring to the given algorithm as MAXLEAF.

\begin{itemize}
    \item Assume (for later contradiction) that MAXLEAF is not correct
    \item thus, some input I exists such that MAXLEAF(I) is not an optimal solution
    \item let OPT(I) be an optimal solution that agrees with MAXLEAF(I) for the most number of steps. Here, "most number of steps" means includes the same vertices in S as MAXLEAF(I) when checking if each leaf is included from the maximum leaf value to the minimum leaf value.
    \item let u be the point where MAXLEAF(I) and OPT(I) first disagree (when looking at the leaves in descending order of value)
    \item due to the definition of MAXLEAF, we know that it always adds the leaf with the maximum value that keeps the solution feasible. Thus, MAXLEAF(I) must select the leaf at u for subset S, while 
    \item let x be the leaf (at point u) that MAXLEAF(I) includes in S
    \item there are 2 options: either OPT(I) doesn't include any leaves after u in S or OPT(I) includes at least one leaf after u
    \item for the first case, OPT'(I) constructed of simply OPT(I) + leaf x is no worse of a solution than OPT(I) and is feasible since MAXLEAF(I) includes x and is feasible => OPT(I) both is and isn't an optimal solution that agrees with MAXLEAF(I) for the most number of steps => contradiction => our assumption that MAXLEAF isn't correct was wrong => MAXLEAF is a correct algorithm
    \item for the second case find leaf y on OPT(I) in the following manner
    \begin{itemize}
        \item if leaf x has any direct siblings (children of x's parent node) that are selected in OPT(I)'s solution S AND are after u in the leaf ordering from highest profit to smallest profit, choose one of them arbitrarily as y
        \item ELSE, if leaf x has any cousins (leaf nodes stemming from x's grandparent) in OPT(I)'s solution AND are after u, choose one of them arbitrarily as y
        \item ELSE, if x's great-grandparent has any descendants that is in OPT(I)'s solution AND are after u, choose one of them arbitrarily as y
        \item continue in this pattern (going up 1 more ancestor from x then looking for descendent leaves in OPT(I) and after u) until a leaf y is found. A leaf y MUST exist since in this case OPT(I) includes at least 1 leaf after u and we know MAXLEAF(I) includes x and is still a feasible solution
    \end{itemize}
    \item now that we have selected y, since y comes after u (and leaf x appears at point u and we were traversing leaves in descending order when comparing MAXLEAF(I) to OPT(I)) we know $p_x \geq p_y$
    \item construct OPT'(I) such that it is exactly the same as OPT(I) except without y, but including x
    \begin{itemize}
        \item OPT'(I) is at least as good of a solution as OPT(I) since $p_x \geq p_y$ so the total profit of OPT'(I) is $\geq$ the total profit of OPT(I)
        \item for OPT(I) to be feasible, it must not exceed the capacity of any of the interior nodes
        \begin{itemize}
            \item when x and y have the same parent, it's the easiest to see that swapping y out and x in to OPT(I) will not change how full any capacities are
            \item when x and y are cousins (same grandparent), swapping y out and x in will keep the same number of leaves under the grandparent, then -1 leaf for y's parent (obviously no feasibility issue), and +1 for x's parent (possibly an issue). This is why we checked for y as a sibling to x FIRST before searching for cousins. Since no sibling to x was in OPT(I), and we know that MAXLEAF(I) was a feasible solution that included x (which OPT(I) didn't include), x's parent is guaranteed to have at least 1 excess capacity int OPT(I)
            \item this pattern continues wtih grandparents and beyond, since we checked for closer related y's first, and MAXLEAF(I) is a feasible solution by definition, we have no feasibility issue since x's parent is guaranteed to have at least 1 excess capacity
        \end{itemize}
        \item since OPT'(I) makes no changes before u, and now agrees with MAXLEAF(I) at u (since it includes x), OPT'(I) agrees with MAXLEAF(I) for more steps than OPT(I)
    \end{itemize}
    \item thus, OPT'(I) is feasible, no worse of a solution than OPT(I), and agrees with MAXLEAF(I) for at least 1 more step than OPT(I)
    \item therefore, OPT(I) is NOT the optimal solution that agrees with MAXLEAF(I) for the most number of steps
    \item CONTRADICTION: we now have OPT(I) both is and is not the optimal solution that agrees with MAXLEAF(I) for the most number of steps
    \item since our only assumption was that MAXLEAF was not a correct algorithm, and we arrived at a contradiction after following logical steps, our assumption MUST be wrong
    \item thus, MAXLEAF is a correct algorithm
    
\end{itemize}

\end{document}
