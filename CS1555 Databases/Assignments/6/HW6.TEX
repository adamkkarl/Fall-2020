\documentclass[a4paper]{article}

%% Language and font encodings
\usepackage[english]{babel}
\usepackage[utf8x]{inputenc}
\usepackage[T1]{fontenc}
\usepackage[normalem]{ulem}

\usepackage[table,xcdraw]{xcolor}

%% Sets page size and margins
\usepackage[a4paper,top=3cm,bottom=2cm,left=1cm,right=1cm,marginparwidth=1.5cm]{geometry}

%% Useful packages
\usepackage{amsmath}
\usepackage{graphicx}
\usepackage[colorinlistoftodos]{todonotes}
\usepackage[colorlinks=true, allcolors=blue]{hyperref}

\title{Assignment 6}
\author{Adam Karl}

\begin{document}
\maketitle

\section{Assignment 6}
\subsection{Synthesis}

Given FDs:

\begin{itemize}
    \item f1: $AB \rightarrow E$
    \item f2: $B \rightarrow DE$
    \item f3: $E \rightarrow D$
    \item f4: $DF \rightarrow A$
    \item f5: $C \rightarrow F$
    \item f6: $DC \rightarrow A$
\end{itemize}

Use Decomposition (IR4) on f2 so that all FDs are in canonical form (only 1 dependent in each FD):

\begin{itemize}
    \item f1: \sout{$AB \rightarrow E$} (from f8 we know $B \rightarrow E$, so A is redundant)
    \item f7: $B \rightarrow D$
    \item f8: $B \rightarrow E$
    \item f3: $E \rightarrow D$
    \item f4: $DF \rightarrow A$
    \item f5: $C \rightarrow F$
    \item f6: $DC \rightarrow A$
\end{itemize}

then

\begin{itemize}
    \item f7: \sout{$B \rightarrow D$} (since $B \rightarrow E$ and $E \rightarrow D$, $B \rightarrow D$ is implied so f7 is redundant)
    \item f8: $B \rightarrow E$
    \item f3: $E \rightarrow D$
    \item f4: $DF \rightarrow A$
    \item f5: $C \rightarrow F$
    \item f6: $DC \rightarrow A$
\end{itemize}

Using Pseudotransitivity, Composition (IR6) on f4: $DF \rightarrow A$ and f5: $C \rightarrow F$ yields $DC \rightarrow A$ which is the same as f6, so f6 is redundant. We remove f6 from consideration.

\begin{itemize}
    \item f8: $B \rightarrow E$
    \item f3: $E \rightarrow D$
    \item f4: $DF \rightarrow A$
    \item f5: $C \rightarrow F$
\end{itemize}

Now that we have the canonical cover, we would group FDs with same determinant, but in this case there aren't any. Therefore we can construct the relations with one FD each.

\begin{itemize}
    \item R1(\underline{B},E)
    \item R2(\underline{E},D)
    \item R3(A,\underline{D},\underline{F})
    \item R4(\underline{C},F)
\end{itemize}

Since we weren't given the key for the original relation, this is our final solution. R1, R2, R3, and R4 are in 3NF and BCNF.



\subsection{Decomposition}
Initial dependencies

\begin{itemize}
    \item $A \rightarrow B$
    \item $B \rightarrow CD$
    \item $A \rightarrow D$
    \item $B \rightarrow C$
    \item $AB \rightarrow CD$
\end{itemize}

Transform all FDs to canonical form (removing duplicates)

\begin{itemize}
    \item $A \rightarrow B$
    \item $B \rightarrow C$
    \item $B \rightarrow D$
    \item $A \rightarrow D$
    \item \sout{$B \rightarrow C$} duplicate
    \item $AB \rightarrow C$
    \item $AB \rightarrow D$
\end{itemize}

Drop extraneous attributes:

\begin{itemize}
    \item $A \rightarrow B$
    \item $B \rightarrow C$
    \item $B \rightarrow D$
    \item $A \rightarrow D$
    \item \sout{$AB \rightarrow C$} already have $B \rightarrow C$
    \item \sout{$AB \rightarrow D$} already have $B \rightarrow D$
\end{itemize}

Drop redundant FDs:

\begin{itemize}
    \item $A \rightarrow B$
    \item $B \rightarrow C$
    \item $B \rightarrow D$
    \item \sout{$A \rightarrow D$} implied by $A \rightarrow B$ and $B \rightarrow D$
\end{itemize}

Therefore the canonical cover of R is:

\begin{itemize}
    \item $A \rightarrow B$
    \item $B \rightarrow C$
    \item $B \rightarrow D$
\end{itemize}

Observation: A does not appear in the right hand side of any FDs, so it must appear in any key of R.

$A+: A \rightarrow AB$ (since $A \rightarrow B$) $\rightarrow ABC$ (since $B \rightarrow C$) $\rightarrow ABCD$ (since $B \rightarrow D$)

We don't need to consider any other combination since any other combination containing A is a super key and not minimal.

R:(\underline{A},B,C,D)

A is the only key we need for R.

\subsection{Table Method}

Our initial table with all columns is:

\begin{table}[htb]
%\centering
\begin{tabular}{|l|l|l|l|l|l|l|l|l|l|l|l|l|l|l|l|}
\hline
   & PID & Len & Wdth & Hght & Wght & OID & ODate & CID & TotPrice & Addr & City & State & Zip & Phone & PQtty \\ \hline
R1 &     &     &      &      &      &     &       &     &          &      &      &       &     &       &       \\ \hline
R2 &     &     &      &      &      &     &       &     &          &      &      &       &     &       &       \\ \hline
R3 &     &     &      &      &      &     &       &     &          &      &      &       &     &       &       \\ \hline
\end{tabular}
\end{table}

Fill in table based on:
\begin{itemize}
    \item R1: (ProductID, Length, Width, Height, Weight, OrderID, OrderDate, CustomerID, TotalPrice)
    \item R2: (CustomerID, Address, City, State, ZipCode, PhoneNumber)
    \item R3: (ProductID, OrderID, ProductQuantity)
\end{itemize}

\begin{table}[htb]
%\centering
\begin{tabular}{|l|l|l|l|l|l|l|l|l|l|l|l|l|l|l|l|}
\hline
   & PID & Len & Wdth & Hght & Wght & OID & ODate & CID & TotPrice & Addr & City & State & Zip & Phone & PQtty \\ \hline
R1 & K   & K   & K    & K    & K    & K   & K     & K   & K        & U    & U    & U     & U   & U     & U     \\ \hline
R2 & U   & U   & U    & U    & U    & U   & U     & K   & U        & K    & K    & K     & K   & K     & U     \\ \hline
R3 & K   & U   & U    & U    & U    & K   & U     & U   & U        & U    & U    & U     & U   & U     & K     \\ \hline
\end{tabular}
\end{table}

\begin{itemize}
    \item FD1: ProductID $\rightarrow$ Length, Width, Height, Weight
\end{itemize}


\begin{table}[htb]
\begin{tabular}{|l|l|l|l|l|l|l|l|l|l|l|l|l|l|l|l|}
\hline
   & PID                                              & Len                      & Wdth                     & Hght                     & Wght                     & OID & ODate & CID & TotPrice & Addr & City & State & Zip & Phone & PQtty \\ \hline
R1 & \cellcolor[HTML]{FFFC9E}{\color[HTML]{FE0000} K} & K                        & K                        & K                        & K                        & K   & K     & K   & K        & U    & U    & U     & U   & U     & U     \\ \hline
R2 & U                                                & U                        & U                        & U                        & U                        & U   & U     & K   & U        & K    & K    & K     & K   & K     & U     \\ \hline
R3 & \cellcolor[HTML]{FFFC9E}{\color[HTML]{FE0000} K} & {\color[HTML]{FE0000} K} & {\color[HTML]{FE0000} K} & {\color[HTML]{FE0000} K} & {\color[HTML]{FE0000} K} & K   & U     & U   & U        & U    & U    & U     & U   & U     & K     \\ \hline
\end{tabular}
\end{table}

\begin{itemize}
    \item FD2: OrderID $\rightarrow$ OrderDate, CustomerID, TotalPrice
\end{itemize}


\begin{tabular}{|l|l|l|l|l|l|l|l|l|l|l|l|l|l|l|l|}
\hline
   & PID                                              & Len                      & Wdth                     & Hght                     & Wght                     & OID                                              & ODate                    & CID                      & TotPrice                 & Addr & City & State & Zip & Phone & PQtty \\ \hline
R1 & \cellcolor[HTML]{FFFFFF}{\color[HTML]{000000} K} & {\color[HTML]{000000} K} & {\color[HTML]{000000} K} & {\color[HTML]{000000} K} & {\color[HTML]{000000} K} & \cellcolor[HTML]{FFFC9E}{\color[HTML]{FE0000} K} & K                        & K                        & K                        & U    & U    & U     & U   & U     & U     \\ \hline
R2 & {\color[HTML]{000000} U}                         & {\color[HTML]{000000} U} & {\color[HTML]{000000} U} & {\color[HTML]{000000} U} & {\color[HTML]{000000} U} & U                                                & U                        & K                        & U                        & K    & K    & K     & K   & K     & U     \\ \hline
R3 & \cellcolor[HTML]{FFFFFF}{\color[HTML]{000000} K} & {\color[HTML]{000000} K} & {\color[HTML]{000000} K} & {\color[HTML]{000000} K} & {\color[HTML]{000000} K} & \cellcolor[HTML]{FFFC9E}{\color[HTML]{FE0000} K} & {\color[HTML]{FE0000} K} & {\color[HTML]{FE0000} K} & {\color[HTML]{FE0000} K} & U    & U    & U     & U   & U     & K     \\ \hline
\end{tabular}

\begin{itemize}
    \item FD3: CustomerID $\rightarrow$ Address, City, State, ZipCode, PhoneNumber
\end{itemize}


\begin{tabular}{|l|
>{\columncolor[HTML]{FFFFFF}}l |
>{\columncolor[HTML]{FFFFFF}}l |
>{\columncolor[HTML]{FFFFFF}}l |
>{\columncolor[HTML]{FFFFFF}}l |
>{\columncolor[HTML]{FFFFFF}}l |
>{\columncolor[HTML]{FFFFFF}}l |
>{\columncolor[HTML]{FFFFFF}}l |
>{\columncolor[HTML]{FFFC9E}}l |
>{\columncolor[HTML]{FFFFFF}}l |
>{\columncolor[HTML]{FFFFFF}}l |
>{\columncolor[HTML]{FFFFFF}}l |l|l|l|l|}
\hline
 &
  {\color[HTML]{333333} PID} &
  {\color[HTML]{333333} Len} &
  {\color[HTML]{333333} Wdth} &
  {\color[HTML]{333333} Hght} &
  {\color[HTML]{333333} Wght} &
  {\color[HTML]{333333} OID} &
  {\color[HTML]{333333} ODate} &
  \cellcolor[HTML]{FFFFFF}{\color[HTML]{333333} CID} &
  {\color[HTML]{333333} TotPrice} &
  {\color[HTML]{333333} Addr} &
  {\color[HTML]{333333} City} &
  State &
  Zip &
  Phone &
  PQtty \\ \hline
R1 &
  {\color[HTML]{333333} K} &
  {\color[HTML]{333333} K} &
  {\color[HTML]{333333} K} &
  {\color[HTML]{333333} K} &
  {\color[HTML]{333333} K} &
  {\color[HTML]{333333} K} &
  {\color[HTML]{333333} K} &
  {\color[HTML]{FE0000} K} &
  {\color[HTML]{333333} K} &
  {\color[HTML]{FE0000} K} &
  {\color[HTML]{FE0000} K} &
  {\color[HTML]{FE0000} K} &
  {\color[HTML]{FE0000} K} &
  {\color[HTML]{FE0000} K} &
  U \\ \hline
R2 &
  {\color[HTML]{333333} U} &
  {\color[HTML]{333333} U} &
  {\color[HTML]{333333} U} &
  {\color[HTML]{333333} U} &
  {\color[HTML]{333333} U} &
  {\color[HTML]{333333} U} &
  {\color[HTML]{333333} U} &
  {\color[HTML]{FE0000} K} &
  {\color[HTML]{333333} U} &
  {\color[HTML]{333333} K} &
  {\color[HTML]{333333} K} &
  K &
  K &
  K &
  U \\ \hline
R3 &
  {\color[HTML]{333333} K} &
  {\color[HTML]{333333} K} &
  {\color[HTML]{333333} K} &
  {\color[HTML]{333333} K} &
  {\color[HTML]{333333} K} &
  {\color[HTML]{333333} K} &
  {\color[HTML]{333333} K} &
  {\color[HTML]{FE0000} K} &
  {\color[HTML]{333333} K} &
  {\color[HTML]{FE0000} K} &
  {\color[HTML]{FE0000} K} &
  {\color[HTML]{FE0000} K} &
  {\color[HTML]{FE0000} K} &
  {\color[HTML]{FE0000} K} &
  K \\ \hline
\end{tabular}


\begin{itemize}
    \item FD4: ProductID, OrderID $\rightarrow$ ProductQuantity
\end{itemize}

\begin{tabular}{|l|
>{\columncolor[HTML]{FFFFFF}}l |
>{\columncolor[HTML]{FFFFFF}}l |
>{\columncolor[HTML]{FFFFFF}}l |
>{\columncolor[HTML]{FFFFFF}}l |
>{\columncolor[HTML]{FFFFFF}}l |
>{\columncolor[HTML]{FFFFFF}}l |
>{\columncolor[HTML]{FFFFFF}}l |
>{\columncolor[HTML]{FFFFFF}}l |
>{\columncolor[HTML]{FFFFFF}}l |
>{\columncolor[HTML]{FFFFFF}}l |
>{\columncolor[HTML]{FFFFFF}}l |l|l|l|l|}
\hline
 & {\color[HTML]{333333} PID} & {\color[HTML]{333333} Len} & {\color[HTML]{333333} Wdth} & {\color[HTML]{333333} Hght} & {\color[HTML]{333333} Wght} & {\color[HTML]{333333} OID} & {\color[HTML]{333333} ODate} & {\color[HTML]{333333} CID} & {\color[HTML]{333333} TotPrice} & {\color[HTML]{333333} Addr} & {\color[HTML]{333333} City} & State & Zip & Phone & PQtty \\ \hline
R1 & \cellcolor[HTML]{FFFC9E}{\color[HTML]{FE0000} K} & {\color[HTML]{333333} K} & {\color[HTML]{333333} K} & {\color[HTML]{333333} K} & {\color[HTML]{333333} K} & \cellcolor[HTML]{FFFC9E}{\color[HTML]{FE0000} K} & {\color[HTML]{333333} K} & {\color[HTML]{333333} K} & {\color[HTML]{333333} K} & {\color[HTML]{333333} K} & {\color[HTML]{333333} K} & \cellcolor[HTML]{FFFFFF}{\color[HTML]{333333} K} & \cellcolor[HTML]{FFFFFF}{\color[HTML]{333333} K} & \cellcolor[HTML]{FFFFFF}{\color[HTML]{333333} K} & {\color[HTML]{FE0000} K} \\ \hline
R2 & {\color[HTML]{333333} U} & {\color[HTML]{333333} U} & {\color[HTML]{333333} U} & {\color[HTML]{333333} U} & {\color[HTML]{333333} U} & {\color[HTML]{333333} U} & {\color[HTML]{333333} U} & {\color[HTML]{333333} K} & {\color[HTML]{333333} U} & {\color[HTML]{333333} K} & {\color[HTML]{333333} K} & \cellcolor[HTML]{FFFFFF}{\color[HTML]{333333} K} & \cellcolor[HTML]{FFFFFF}{\color[HTML]{333333} K} & \cellcolor[HTML]{FFFFFF}{\color[HTML]{333333} K} & U \\ \hline
R3 & \cellcolor[HTML]{FFFC9E}{\color[HTML]{FE0000} K} & {\color[HTML]{333333} K} & {\color[HTML]{333333} K} & {\color[HTML]{333333} K} & {\color[HTML]{333333} K} & \cellcolor[HTML]{FFFC9E}{\color[HTML]{FE0000} K} & {\color[HTML]{333333} K} & {\color[HTML]{333333} K} & {\color[HTML]{333333} K} & {\color[HTML]{333333} K} & {\color[HTML]{333333} K} & \cellcolor[HTML]{FFFFFF}{\color[HTML]{333333} K} & \cellcolor[HTML]{FFFFFF}{\color[HTML]{333333} K} & \cellcolor[HTML]{FFFFFF}{\color[HTML]{333333} K} & K \\ \hline
\end{tabular}

Since row 1 (also row 3) consists of all known values, \textbf{the decomposition is lossless.} However, we must still check if all dependencies are preserved. To do this, we will find the canonical FDs and check if they span across tables.

Our given FDs are:

\begin{itemize}
    \item ProductID $\rightarrow$ Length, Width, Height, Weight
    \item OrderID $\rightarrow$ OrderDate, CustomerID, TotalPrice
    \item CustomerID $\rightarrow$ Address, City, State, ZipCode, PhoneNumber
    \item ProductID, OrderID $\rightarrow$ ProductQuantity
\end{itemize}

Transform all FDs to canonical form:

\begin{enumerate}
    \item ProductID $\rightarrow$ Length
    \item ProductID $\rightarrow$ Width
    \item ProductID $\rightarrow$ Height
    \item ProductID $\rightarrow$ Weight
    \item OrderID $\rightarrow$ OrderDate
    \item OrderID $\rightarrow$ CustomerID
    \item OrderID $\rightarrow$ TotalPrice
    \item CustomerID $\rightarrow$ Address
    \item CustomerID $\rightarrow$ City
    \item CustomerID $\rightarrow$ State
    \item CustomerID $\rightarrow$ ZipCode
    \item CustomerID $\rightarrow$ PhoneNumber
    \item ProductID, OrderID $\rightarrow$ ProductQuantity
\end{enumerate}

Remember the given decomposition relations are:

\begin{itemize}
    \item R1: (\underline{ProductID}, Length, Width, Height, Weight, \underline{OrderID}, OrderDate, CustomerID, TotalPrice)
    \item R2: (\underline{CustomerID}, Address, City, State, ZipCode, PhoneNumber)
    \item R3: (\underline{ProductID}, \underline{OrderID}, ProductQuantity)
\end{itemize}

When considering these canonical dependencies to the given relations:

\begin{itemize}
    \item the key for FDs 1-7 appear both in R1 and R3, but only have the dependents in R1.
    \item the key for FDs 8-12 appear in R2 along with each dependent. No issue
    \item the key for FD 13 appears both in R1 and R3, but only has the dependent in R3.
\end{itemize}

Since canonical FDs 1-7 and 13 span across tables, this decomposition is non-dependency preserving.

Since we have shown this decomposition is lossless but non-dependency preserving, \textbf{it is an ugly decomposition.}

(Note: simply removing R3 would be sufficient for a good decomposition)
\end{document}
