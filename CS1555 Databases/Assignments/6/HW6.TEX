\documentclass[a4paper]{article}

%% Language and font encodings
\usepackage[english]{babel}
\usepackage[utf8x]{inputenc}
\usepackage[T1]{fontenc}
\usepackage[normalem]{ulem}
\usepackage[table,xcdraw]{xcolor}

%% Sets page size and margins
\usepackage[a4paper,top=3cm,bottom=2cm,left=1cm,right=1cm,marginparwidth=1.5cm]{geometry}

%% Useful packages
\usepackage{amsmath}
\usepackage{graphicx}
\usepackage[colorinlistoftodos]{todonotes}
\usepackage[colorlinks=true, allcolors=blue]{hyperref}


\title{Homework 12}
\author{Adam Karl}

\begin{document}
\maketitle

\section{Problem 37 (6 points)}
\subsection{Synthesis}

\subsection{Decomposition}
Initial dependencies

\begin{itemize}
    \item $A \rightarrow B$
    \item $B \rightarrow CD$
    \item $A \rightarrow D$
    \item $B \rightarrow C$
    \item $AB \rightarrow CD$
\end{itemize}

Transform all FDs to canonical form (removing duplicates)

\begin{itemize}
    \item $A \rightarrow B$
    \item $B \rightarrow C$
    \item $B \rightarrow D$
    \item $A \rightarrow D$
    \item \sout{$B \rightarrow C$} duplicate
    \item $AB \rightarrow C$
    \item $AB \rightarrow D$
\end{itemize}

Drop extraneous attributes

\begin{itemize}
    \item $A \rightarrow B$
    \item $B \rightarrow C$
    \item $B \rightarrow D$
    \item $A \rightarrow D$
    \item \sout{$AB \rightarrow C$} already have $B \rightarrow C$
    \item \sout{$AB \rightarrow D$} already have $B \rightarrow D$
\end{itemize}

Drop redundant FDs
\begin{itemize}
    \item $A \rightarrow B$
    \item $B \rightarrow C$
    \item $B \rightarrow D$
    \item \sout{$A \rightarrow D$} implied by $A \rightarrow B$ and $B \rightarrow D$
\end{itemize}

Therefore the canonical cover of R is:

\begin{itemize}
    \item $A \rightarrow B$
    \item $B \rightarrow C$
    \item $B \rightarrow D$
\end{itemize}

Observation: A does not appear in the right hand side of any FDs, so it must appear in any key of R.

$A+: A \rightarrow AB$ (since $A \rightarrow B$) $\rightarrow ABC$ (since $B \rightarrow C$) $\rightarrow ABCD$ (since $B \rightarrow D$)

We don't need to consider any other combination since any other combination containing A is a super key and not minimal.

R:(\underline{A},B,C,D)

A is the only key we need for R.

\subsection{Table Method}
\begin{table}[htb]
%\centering
\begin{tabular}{|l|l|l|l|l|l|l|l|l|l|l|l|l|l|l|l|}
\hline
   & PID & Len & Wdth & Hght & Wght & OID & ODate & CID & TotPrice & Addr & City & State & Zip & Phone & PQtty \\ \hline
R1 &     &     &      &      &      &     &       &     &          &      &      &       &     &       &       \\ \hline
R2 &     &     &      &      &      &     &       &     &          &      &      &       &     &       &       \\ \hline
R3 &     &     &      &      &      &     &       &     &          &      &      &       &     &       &       \\ \hline
\end{tabular}
\end{table}

Fill in table based on:
\begin{itemize}
    \item R1: (ProductID, Length, Width, Height, Weight, OrderID, OrderDate, CustomerID, Total-Price)
    \item R2: (CustomerID, Address, City, State, ZipCode, PhoneNumber)
    \item R3: (ProductID, OrderID, ProductQuantity)
\end{itemize}

\begin{table}[htb]
%\centering
\begin{tabular}{|l|l|l|l|l|l|l|l|l|l|l|l|l|l|l|l|}
\hline
   & PID & Len & Wdth & Hght & Wght & OID & ODate & CID & TotPrice & Addr & City & State & Zip & Phone & PQtty \\ \hline
R1 & K   & K   & K    & K    & K    & K   & K     & K   & K        & U    & U    & U     & U   & U     & U     \\ \hline
R2 & U   & U   & U    & U    & U    & U   & U     & K   & U        & K    & K    & K     & K   & K     & U     \\ \hline
R3 & K   & U   & U    & U    & U    & K   & U     & U   & U        & U    & U    & U     & U   & U     & K     \\ \hline
\end{tabular}
\end{table}

Consider
\begin{itemize}
    \item FD1: ProductID $\rightarrow$ Length, Width, Height, Weight
\end{itemize}


\begin{table}[htb]
\begin{tabular}{|l|l|l|l|l|l|l|l|l|l|l|l|l|l|l|l|}
\hline
   & PID                                              & Len                      & Wdth                     & Hght                     & Wght                     & OID & ODate & CID & TotPrice & Addr & City & State & Zip & Phone & PQtty \\ \hline
R1 & \cellcolor[HTML]{FFFC9E}{\color[HTML]{FE0000} K} & K                        & K                        & K                        & K                        & K   & K     & K   & K        & U    & U    & U     & U   & U     & U     \\ \hline
R2 & U                                                & U                        & U                        & U                        & U                        & U   & U     & K   & U        & K    & K    & K     & K   & K     & U     \\ \hline
R3 & \cellcolor[HTML]{FFFC9E}{\color[HTML]{FE0000} K} & {\color[HTML]{FE0000} K} & {\color[HTML]{FE0000} K} & {\color[HTML]{FE0000} K} & {\color[HTML]{FE0000} K} & K   & U     & U   & U        & U    & U    & U     & U   & U     & K     \\ \hline
\end{tabular}
\end{table}

Consider
\begin{itemize}
    \item FD2: OrderID $\rightarrow$ OrderDate, CustomerID, TotalPrice
\end{itemize}


\begin{tabular}{|l|l|l|l|l|l|l|l|l|l|l|l|l|l|l|l|}
\hline
   & PID                                              & Len                      & Wdth                     & Hght                     & Wght                     & OID                                              & ODate                    & CID                      & TotPrice                 & Addr & City & State & Zip & Phone & PQtty \\ \hline
R1 & \cellcolor[HTML]{FFFFFF}{\color[HTML]{000000} K} & {\color[HTML]{000000} K} & {\color[HTML]{000000} K} & {\color[HTML]{000000} K} & {\color[HTML]{000000} K} & \cellcolor[HTML]{FFFC9E}{\color[HTML]{FE0000} K} & K                        & K                        & K                        & U    & U    & U     & U   & U     & U     \\ \hline
R2 & {\color[HTML]{000000} U}                         & {\color[HTML]{000000} U} & {\color[HTML]{000000} U} & {\color[HTML]{000000} U} & {\color[HTML]{000000} U} & U                                                & U                        & K                        & U                        & K    & K    & K     & K   & K     & U     \\ \hline
R3 & \cellcolor[HTML]{FFFFFF}{\color[HTML]{000000} K} & {\color[HTML]{000000} K} & {\color[HTML]{000000} K} & {\color[HTML]{000000} K} & {\color[HTML]{000000} K} & \cellcolor[HTML]{FFFC9E}{\color[HTML]{FE0000} K} & {\color[HTML]{FE0000} K} & {\color[HTML]{FE0000} K} & {\color[HTML]{FE0000} K} & U    & U    & U     & U   & U     & K     \\ \hline
\end{tabular}

Consider
\begin{itemize}
    \item FD3: CustomerID $\rightarrow$ Address, City, State, ZipCode, PhoneNumber
\end{itemize}


\begin{tabular}{|l|
>{\columncolor[HTML]{FFFFFF}}l |
>{\columncolor[HTML]{FFFFFF}}l |
>{\columncolor[HTML]{FFFFFF}}l |
>{\columncolor[HTML]{FFFFFF}}l |
>{\columncolor[HTML]{FFFFFF}}l |
>{\columncolor[HTML]{FFFFFF}}l |
>{\columncolor[HTML]{FFFFFF}}l |
>{\columncolor[HTML]{FFFC9E}}l |
>{\columncolor[HTML]{FFFFFF}}l |
>{\columncolor[HTML]{FFFFFF}}l |
>{\columncolor[HTML]{FFFFFF}}l |l|l|l|l|}
\hline
 &
  {\color[HTML]{333333} PID} &
  {\color[HTML]{333333} Len} &
  {\color[HTML]{333333} Wdth} &
  {\color[HTML]{333333} Hght} &
  {\color[HTML]{333333} Wght} &
  {\color[HTML]{333333} OID} &
  {\color[HTML]{333333} ODate} &
  \cellcolor[HTML]{FFFFFF}{\color[HTML]{333333} CID} &
  {\color[HTML]{333333} TotPrice} &
  {\color[HTML]{333333} Addr} &
  {\color[HTML]{333333} City} &
  State &
  Zip &
  Phone &
  PQtty \\ \hline
R1 &
  {\color[HTML]{333333} K} &
  {\color[HTML]{333333} K} &
  {\color[HTML]{333333} K} &
  {\color[HTML]{333333} K} &
  {\color[HTML]{333333} K} &
  {\color[HTML]{333333} K} &
  {\color[HTML]{333333} K} &
  {\color[HTML]{FE0000} K} &
  {\color[HTML]{333333} K} &
  {\color[HTML]{FE0000} K} &
  {\color[HTML]{FE0000} K} &
  {\color[HTML]{FE0000} K} &
  {\color[HTML]{FE0000} K} &
  {\color[HTML]{FE0000} K} &
  U \\ \hline
R2 &
  {\color[HTML]{333333} U} &
  {\color[HTML]{333333} U} &
  {\color[HTML]{333333} U} &
  {\color[HTML]{333333} U} &
  {\color[HTML]{333333} U} &
  {\color[HTML]{333333} U} &
  {\color[HTML]{333333} U} &
  {\color[HTML]{FE0000} K} &
  {\color[HTML]{333333} U} &
  {\color[HTML]{333333} K} &
  {\color[HTML]{333333} K} &
  K &
  K &
  K &
  U \\ \hline
R3 &
  {\color[HTML]{333333} K} &
  {\color[HTML]{333333} K} &
  {\color[HTML]{333333} K} &
  {\color[HTML]{333333} K} &
  {\color[HTML]{333333} K} &
  {\color[HTML]{333333} K} &
  {\color[HTML]{333333} K} &
  {\color[HTML]{FE0000} K} &
  {\color[HTML]{333333} K} &
  {\color[HTML]{FE0000} K} &
  {\color[HTML]{FE0000} K} &
  {\color[HTML]{FE0000} K} &
  {\color[HTML]{FE0000} K} &
  {\color[HTML]{FE0000} K} &
  K \\ \hline
\end{tabular}

Consider
\begin{itemize}
    \item FD4: ProductID, OrderID $\rightarrow$ ProductQuantity
\end{itemize}

\begin{tabular}{|l|
>{\columncolor[HTML]{FFFFFF}}l |
>{\columncolor[HTML]{FFFFFF}}l |
>{\columncolor[HTML]{FFFFFF}}l |
>{\columncolor[HTML]{FFFFFF}}l |
>{\columncolor[HTML]{FFFFFF}}l |
>{\columncolor[HTML]{FFFFFF}}l |
>{\columncolor[HTML]{FFFFFF}}l |
>{\columncolor[HTML]{FFFFFF}}l |
>{\columncolor[HTML]{FFFFFF}}l |
>{\columncolor[HTML]{FFFFFF}}l |
>{\columncolor[HTML]{FFFFFF}}l |l|l|l|l|}
\hline
 & {\color[HTML]{333333} PID} & {\color[HTML]{333333} Len} & {\color[HTML]{333333} Wdth} & {\color[HTML]{333333} Hght} & {\color[HTML]{333333} Wght} & {\color[HTML]{333333} OID} & {\color[HTML]{333333} ODate} & {\color[HTML]{333333} CID} & {\color[HTML]{333333} TotPrice} & {\color[HTML]{333333} Addr} & {\color[HTML]{333333} City} & State & Zip & Phone & PQtty \\ \hline
R1 & \cellcolor[HTML]{FFFC9E}{\color[HTML]{FE0000} K} & {\color[HTML]{333333} K} & {\color[HTML]{333333} K} & {\color[HTML]{333333} K} & {\color[HTML]{333333} K} & \cellcolor[HTML]{FFFC9E}{\color[HTML]{FE0000} K} & {\color[HTML]{333333} K} & {\color[HTML]{333333} K} & {\color[HTML]{333333} K} & {\color[HTML]{333333} K} & {\color[HTML]{333333} K} & \cellcolor[HTML]{FFFFFF}{\color[HTML]{333333} K} & \cellcolor[HTML]{FFFFFF}{\color[HTML]{333333} K} & \cellcolor[HTML]{FFFFFF}{\color[HTML]{333333} K} & {\color[HTML]{FE0000} K} \\ \hline
R2 & {\color[HTML]{333333} U} & {\color[HTML]{333333} U} & {\color[HTML]{333333} U} & {\color[HTML]{333333} U} & {\color[HTML]{333333} U} & {\color[HTML]{333333} U} & {\color[HTML]{333333} U} & {\color[HTML]{333333} K} & {\color[HTML]{333333} U} & {\color[HTML]{333333} K} & {\color[HTML]{333333} K} & \cellcolor[HTML]{FFFFFF}{\color[HTML]{333333} K} & \cellcolor[HTML]{FFFFFF}{\color[HTML]{333333} K} & \cellcolor[HTML]{FFFFFF}{\color[HTML]{333333} K} & U \\ \hline
R3 & \cellcolor[HTML]{FFFC9E}{\color[HTML]{FE0000} K} & {\color[HTML]{333333} K} & {\color[HTML]{333333} K} & {\color[HTML]{333333} K} & {\color[HTML]{333333} K} & \cellcolor[HTML]{FFFC9E}{\color[HTML]{FE0000} K} & {\color[HTML]{333333} K} & {\color[HTML]{333333} K} & {\color[HTML]{333333} K} & {\color[HTML]{333333} K} & {\color[HTML]{333333} K} & \cellcolor[HTML]{FFFFFF}{\color[HTML]{333333} K} & \cellcolor[HTML]{FFFFFF}{\color[HTML]{333333} K} & \cellcolor[HTML]{FFFFFF}{\color[HTML]{333333} K} & K \\ \hline
\end{tabular}

Since row 1 (also row 3) consists of all known values, \textbf{the decomposition is lossless.}

\end{document}
